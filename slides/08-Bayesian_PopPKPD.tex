\section{Bayesian Population Pharmacodynamic Modeling}

\subsection{Recommended References}
\begin{frame}{Bayesian Population Pharmacokinetic Modeling - Recommended References}
    \begin{vfilleditems}
        \item \textcite{Gabrielsson2006PKPDbook}:
        \begin{vfilleditems}
            \item Chapter 1: General Principles
            \item Chapter 3: Pharmacodynamic Concepts
        \end{vfilleditems}
        \item \textcite{Owen2014PKPDbook}:
        \begin{vfilleditems}
            \item Chapter 10: PK/PD Models
        \end{vfilleditems}
        \item \textcite{Bonate2011PKPDbook}:
        \begin{vfilleditems}
            \item Chapter 10: Bayesian Modeling regression
        \end{vfilleditems}
        \item \textcite{margossian2022torsten}
    \end{vfilleditems}
\end{frame}

\subsection{Pharmacodynamics}
\begin{frame}{Pharmacodynamics}
    \begin{defn}[Pharmacodynamics]
        \begin{quotation}
            Pharmacodynamics can be defined as the study of the time course of the
            biological effects of drugs, the relationship of the effects to drug exposure,
            and the mechanisms of drug action.
        \end{quotation}
        \vfill \vfill
        \textcite[199]{Gabrielsson2006PKPDbook}
    \end{defn}
\end{frame}

\begin{frame}{Pharmacodynamics}
    Pharmacodynamics is generally represented as \textbf{``PD'' compartments} in a model.
    \vfill
    They can be either:
    \begin{vfilleditems}
        \item \textbf{Pharmacodynamic} (PD) models with only PD compartments
        \item \textbf{Phamacokynetic-Pharmacodynamic} (PKPD) models with both PK compartments and PD compartments
    \end{vfilleditems}
\end{frame}

\subsection{Compartment Models}